
% RUN in terminal (without bibliography):
% pdflatex -output-directory=/Users/salvatorpes/Desktop/LFEUI/text/proposta /Users/salvatorpes/Desktop/LFEUI/text/proposta/prop.tex

% Limpar:
% latexmk -c

\documentclass{article}

\author{
    \begin{tabular}{rl}
        Estêvão Gomes (ist1102650) & Sofia Nunes (ist1102633) \\
        Pedro Curvo (ist1102716) & Salvador Torpes (ist1102474)
    \end{tabular}
}

\usepackage[utf8]{inputenc}
\usepackage[english]{babel}
\usepackage[letterpaper,top=10mm,bottom=15mm,left=10mm,right=10mm,marginparwidth=1.75cm]{geometry}
\usepackage{multicol}
\usepackage{graphicx}
\usepackage{subcaption}
\usepackage{tabularx}
\usepackage{booktabs}
\usepackage{array}
\usepackage{makecell}
\usepackage{titlesec}
\usepackage{multirow}
\usepackage{amsmath}
\usepackage{makecell}
\usepackage{url}
\usepackage{csquotes}
\usepackage{caption}
\usepackage{enumitem}
\usepackage{textcomp}
\usepackage{pdflscape}
\usepackage{makeidx}
% \usepackage{tocbibind}
\providecommand{\tightlist}{\relax}
\usepackage{tocloft}
\renewcommand{\cftsecindent}{0em}
\renewcommand{\cftsubsecindent}{1em}
\renewcommand{\cftsecfont}{\bfseries}
\renewcommand{\cftsubsecfont}{\itshape}
\setlength{\cftsubsecnumwidth}{0em}

\usepackage[version=4]{mhchem}
\usepackage{hyperref} % Remove "pdftex" option here
\usepackage{float}
\usepackage{fancyhdr}
\usepackage{ragged2e}
\usepackage{xkeyval}
%\usepackage{minted}
%\usemintedstyle{manni}
\usepackage{listings}
\usepackage{amssymb}


\usepackage{xcolor}
\usepackage{tikz}

\usetikzlibrary{positioning}
\usetikzlibrary{positioning, arrows.meta}
\usepackage{adjustbox}
\usepackage{sidecap}
\usepackage{graphicx}

\usepackage{tikz-3dplot}
\usepackage{pgfplots}
\usetikzlibrary{calc, 3d, arrows}



\usetikzlibrary{shapes.geometric, arrows}


\lstset{
    language=Python,
    basicstyle=\ttfamily,
    keywordstyle=\color{blue},
    commentstyle=\color{gray},
    stringstyle=\color{orange},
    numbers=left,
    numberstyle=\tiny,
    numbersep=5pt,
    showspaces=false,
    showstringspaces=false,
    breaklines=true,
    frame=tb,
    framexleftmargin=2em,
    xleftmargin=2em,
}


%\usepackage{fontspec}

%\setmonofont{Fira Code}

\fancyhf{}
\cfoot{\thepage}
\fancyhf{} % Clear all header and footer fields
\renewcommand{\headrulewidth}{0pt} % Remove the header rule line
\cfoot{\thepage} % Set the page number in the center of the footer

\pagestyle{fancy} % Apply the fancy page style

\setlength\columnsep{20pt}

\renewcommand{\familydefault}{\sfdefault}

\newenvironment{Figure}
  {\par\medskip\noindent\minipage{\linewidth}}
  {\endminipage\par\medskip}

\makeatletter
\newenvironment{figurehere}
{\def\@captype{figure}}
{}
\makeatother

\hypersetup{
  colorlinks,
  linkcolor=blue,
  anchorcolor=black,
  citecolor=cyan,
  filecolor=cyan,
  menucolor=cyan,
  urlcolor=cyan,
  bookmarksopen=true,
  bookmarksnumbered=true
}

\makeindex


\title{\vspace{-13mm}\includegraphics[width=15mm,scale=3]{../images/IST_Logo.png}\\ \vspace{5mm}
LFEUI - Logbook 1\vspace{-5mm}}
\date{30 November 2023}

\usepackage{sansmathfonts}
\usepackage[T1]{fontenc}
\usepackage[OT1]{fontenc}

\titleformat{\section}{\normalfont\large\bfseries}{\thesection}{1em}{}



\usepackage[style=numeric]{biblatex} % Choose your desired citation style
\addbibresource{../references/prop.bib} % Specify your .bib file

\begin{document}

\renewcommand{\arraystretch}{1.5}
\setlength{\columnseprule}{0.4pt}
\tdplotsetmaincoords{70}{110} % Set the viewing angle
\newcolumntype{M}[1]{>{\centering\arraybackslash\vspace{#1}}m{0.5\linewidth}<{\vspace{#1}}}
\newcolumntype{C}[2]{>{\centering\arraybackslash\vspace{#1}\rule{0pt}{#1}\hspace{0pt}}m{#2}}
\newcolumntype{w}[1]{>{\centering\arraybackslash}m{#1}}

\renewcommand*\familydefault{\sfdefault} %% Only if the base font of the document is to be sans serif

\maketitle

\vspace{-5mm}

\hrulefill

% \begin{multicols}{2}

\vspace*{1cm}

On the 30\textsuperscript{th} of November, the four members of our group went to the second meeting with professors Rodrigo Mateus, Rui Silva and Norberto Catarino at CTN Loures. Our goal was to discuss the work we will be doing in the following session at the same laboratory.
We arrived at 9:30 AM and started discussing the questions that we had been requested to answer the previous session.

\paragraph*{}

The professors turned on the accelerator and went through the different components of its layout and experiment setup. We focused on the detector part and we were able to see both the samples and the silicon detectors next to it.

\paragraph*{}

The rest of the session was followed by a discussion about multiple theoretical and experimental aspects of the project. We discussed the following topics:

\begin{enumerate}
  \item The structure of the tandem 2.5 MeV accelerator and both it's acceleration tubes. They allow the proton beam to reach the target with a kinetic energy of 5 MeV instead of 2.5 MeV (only one tube);
  \item The radiation at the laboratory - we were told that the radiation levels are very low unless we were very close to the accelerator.
  \item The vacuum inside the tubes - we estimated that, at a pressure of $10^{-7}$ mbar (the pressure inside the tubes), there are approximately $10^9$ particles per cubic centimeter. This is a good vacuum meaning that the protons won't collide with other particles and will be able to reach the target without losing energy because the mean free path is longer than the accelerator;
  \item The equation that we aim to study: a proton collides with $^7$Li and produces $^8$Be, which then decays into two alpha particles. The decay process is not detectable, so we will only be able to detect the alpha particles:
  
  \begin{equation}
    \ce{^7Li + ^1H -> ^8Be -> 2\alpha}
  \end{equation}
  
  \item The importance of the electromagnets used to deviate the trajectory of the proton beam. We understood that deflection angle is related to the mass and charge of the particle, therefore different particles will have different trajectories, meaning we can isolate the protons on the incident beam;
  \item Samples used: in the next session, we need to collect the radiation spectrum of two different Lithium-7 samples: one of them is a glass sample composed of silicon, boron and lithium and the other one is a implant sample, which is a thin layer of lithium implanted in a silicon substrate;
  \item Finally, we discussed the expected results for both samples - phenomena such as Rutherford Backscattering, Elastic Scattering and profile of the energy spectrum of the alpha particles were mentioned.
\end{enumerate}

We left the laboratory at 12:00 PM.

% \end{multicols}

\end{document}