% RUN in terminal (without bibliography):
% pdflatex -output-directory=/Users/salvatorpes/Desktop/LFEUI/text/proposta /Users/salvatorpes/Desktop/LFEUI/text/proposta/prop.tex
% pdflatex -output-directory=/Users/estevao/Desktop/Universidade/LFEUI/text/paper /Users/estevao/Desktop/Universidade/LFEUI/text/paper/paper.tex

\documentclass{article}

\author{
    \begin{tabular}{rl}
        Estêvão Gomes (ist1102650) & Sofia Nunes (ist1102633) \\
        Pedro Curvo (ist1102716) & Salvador Torpes (ist1102474)
    \end{tabular}
}

\usepackage[utf8]{inputenc}
\usepackage[english]{babel}
\usepackage[letterpaper,top=6mm,bottom=15mm,left=6mm,right=6mm,marginparwidth=1.55cm]{geometry}
\usepackage{multicol}
\usepackage{graphicx}
\usepackage{subcaption}
\usepackage{tabularx}
\usepackage{booktabs}
\usepackage{array}
\usepackage{makecell}
\usepackage{titlesec}
\usepackage{multirow}
\usepackage{amsmath}
\usepackage{makecell}
\usepackage{url}
\usepackage{csquotes}
\usepackage{caption}
\usepackage{enumitem}
\usepackage{textcomp}
\usepackage{pdflscape}
\usepackage{makeidx}
\usepackage{mathtools}
% \usepackage{tocbibind}
\providecommand{\tightlist}{\relax}
\usepackage{tocloft}
\renewcommand{\cftsecindent}{0em}
\renewcommand{\cftsubsecindent}{1em}
\renewcommand{\cftsecfont}{\bfseries}
\renewcommand{\cftsubsecfont}{\itshape}
\setlength{\cftsubsecnumwidth}{0em}

\usepackage[version=4]{mhchem}
\usepackage{hyperref} % Remove "pdftex" option here
\usepackage{float}
\usepackage{fancyhdr}
\usepackage{ragged2e}
\usepackage{xkeyval}
%\usepackage{minted}
%\usemintedstyle{manni}
\usepackage{listings}
\usepackage{amssymb}


\usepackage{xcolor}
\usepackage{tikz}

\usetikzlibrary{positioning}
\usetikzlibrary{positioning, arrows.meta}
\usepackage{adjustbox}
\usepackage{sidecap}
\usepackage{graphicx}

\usepackage{tikz-3dplot}
\usepackage{pgfplots}
\usetikzlibrary{calc, 3d, arrows}



\usetikzlibrary{shapes.geometric, arrows}


\lstset{
    language=Python,
    basicstyle=\ttfamily,
    keywordstyle=\color{blue},
    commentstyle=\color{gray},
    stringstyle=\color{orange},
    numbers=left,
    numberstyle=\tiny,
    numbersep=5pt,
    showspaces=false,
    showstringspaces=false,
    breaklines=true,
    frame=tb,
    framexleftmargin=2em,
    xleftmargin=2em,
}


%\usepackage{fontspec}

%\setmonofont{Fira Code}

\fancyhf{}
\cfoot{\thepage}
\fancyhf{} % Clear all header and footer fields
\renewcommand{\headrulewidth}{0pt} % Remove the header rule line
\cfoot{\thepage} % Set the page number in the center of the footer

\pagestyle{fancy} % Apply the fancy page style

\setlength\columnsep{20pt}

\renewcommand{\familydefault}{\sfdefault}

\newenvironment{Figure}
  {\par\medskip\noindent\minipage{\linewidth}}
  {\endminipage\par\medskip}

\makeatletter
\newenvironment{figurehere}
{\def\@captype{figure}}
{}
\makeatother

\hypersetup{
  colorlinks,
  linkcolor=blue,
  anchorcolor=black,
  citecolor=cyan,
  filecolor=cyan,
  menucolor=cyan,
  urlcolor=cyan,
  bookmarksopen=true,
  bookmarksnumbered=true
}

\makeindex


\title{\vspace{-13mm}\includegraphics[width=15mm,scale=3]{../images/IST_Logo.png}
% \\ \vspace{1mm} LFEUI \vspace{1mm}
\\ {\fontsize{24}{16}Detection of Li in Technological Materials through Nuclear Reactions} \vspace{-5mm}}
\date{December 2023 - January 2024}

\usepackage{sansmathfonts}
\usepackage[T1]{fontenc}
\usepackage[OT1]{fontenc}

\titleformat{\section}{\normalfont\large\bfseries}{\thesection}{1em}{}



\usepackage[backend=biber, style=numeric]{biblatex}
\addbibresource{paper.bib}

\setlist[enumerate]{itemsep=3pt}

\begin{document}

\renewcommand{\arraystretch}{1.5}
\setlength{\columnseprule}{0.4pt}
\tdplotsetmaincoords{70}{110} % Set the viewing angle
\newcolumntype{M}[1]{>{\centering\arraybackslash\vspace{#1}}m{0.5\linewidth}<{\vspace{#1}}}
\newcolumntype{C}[2]{>{\centering\arraybackslash\vspace{#1}\rule{0pt}{#1}\hspace{0pt}}m{#2}}
\newcolumntype{w}[1]{>{\centering\arraybackslash}m{#1}}

\renewcommand*\familydefault{\sfdefault} %% Only if the base font of the document is to be sans serif

\maketitle

\vspace{-8mm}


\hrulefill

% \begin{center}
%   \textbf{\Large Abstract}
% \end{center}

\vspace{-0.1cm}
\begin{abstract}
  \par
  The following paper describes a scientific research on the ability to detect the presence of lithium ($\ce{^7Li}$) through the byproducts of nuclear reactions.
  The research was conducted in 3 different samples: lithium fluoride ($\ce{LiF}$), lithium aluminate ($\ce{LiAlO2}$) and an implanted sample.
  Energy spectra were obtained for each sample with a tandem accelerator and $1.3$ MeV protons.
  Considering the theoretical values for the nuclear reaction of $\ce{^7Li}$, we were able to identify the presence of lithium in all samples.
  Some other phenomena were observed, such as back-scattering, pile-up and energy profiling.
  Finally, using the information on each spectrum, we estimated the quantity of lithium in each sample.
\end{abstract}
\vspace{-0.1cm}

\hrulefill

\begin{multicols}{2}
\small
\section{Main Goal}
    \label{sec:maingoal}

Our main goal is to develop a scientific research project alongside 3 senior investigators from CTN (Centro Tecnológico e Nuclear), Rodrigo Mateus, Rui Silva and Norberto Catarino.
Our project will be focused on the detection of Li in technological materials through nuclear reactions. The main motivation for this project is to study and develop a detection technique for elements such as Li with high sensitivity.

\section{Scientific Background}
    \label{sec:scientificbackground}

The development of this scientific procedure is based on nuclear reactions. We have an accelerator that allows protons to reach high energyies in order to collide with the sample whose composition we are interested in studying. The protons will undergo a nuclear reaction with the nuclei in the samples and byproducts of these reaction will be released. These are charged particles that we can detect and quantify through their energy spectrum.

\subsection{Nuclear Reaction}
    \label{sec:nuclearreactions}

The main nuclear reaction we are interested in is between $\ce{^7Li}$ and an incoming proton from the beam:

\begin{equation}
  \ce{^7Li + ^1H -> ^8Be -> 2\alpha}
\label{eq:reactionLi}
\end{equation}

Before the emission of the reaction products, the reaction leads to an intermediate state, a $\ce{^8Be}$ compound nucleus, which then rapidly decays into two alpha particles that we can detect.
An alpha particle is a helium nucleus, composed of two protons and two neutrons. % toda a gente sabe o que é uma particula alpha valhamedeus

% \paragraph{}

% Some of the samples we are going to work with also include boron and according to the beam energy we use it is possible that the protons lead to a nuclear reaction with the boron atoms in the samples. Therefore, the $\ce{^{11}B}$ reaction with a proton is as follows:

% \begin{equation}
%   \ce{ ^{11}B + ^1H -> ^{12}C -> ^8Be + \alpha}
% \label{eq:reactionB}
% \end{equation}

% The reaction first leads to the formation of a $\ce{^{12}C}$ compound nucleus which decays into one alpha particle and a $\ce{^8Be}$ nucleus. The $\alpha$ particle can be detected with more or less energy according to the energy level on which the $\ce{^8Be}$ nucleus is emitted. If the latter is on the fundamental state, the $\alpha$ particle will have more energy and if it is on a higher state, the $\alpha$ particle will have less energy.

\subsection{Energy Peaks and Energy Profile}
    \label{sec:energy_profile_peak}

Regarding the phenomena expected in the collected energy spectrum, the primary aspect involves the energy values.
Both the alpha particles and other byproducts are emitted with distinct energies immediately following the nuclear reaction.
These energies can be calculated based on the reaction and the beam's energy.
The computation process will be detailed in section \ref{sec:theoreticalvalues}.

\paragraph{}

Firstly, we expect to see intensity peaks on the spectrum on these specific energies.
However, we must take into account the fact that before arriving at the detector, the emitted particles have to go through the other atoms on the sample.
As an incident proton comes in, it might react with another proton from the material that sits on its surface but it can also react with one inside it.
The byproducts produced inside the sample initially have the energy we compute but do lose part of it as they collide with other particles before reaching the detector.
Basically, we must consider the sample's thickness. The bigger the depth at which the reaction takes place, the smaller the energy with whom they are detected because more of it gets dissipated.
Therefore, this will result in an energy profile - the energy peak will be gradually preceded by less intense peaks due to particles that lost energy on their way to the detector.
It is less likely for the particles that are emitted inside the sample to reach the detector than for those who were emitted on its surface since on each collision there is a chance that it's direction might change.
This is why the intensity on the energy profile has a steady increase before hitting a maximum value at the energy with whom the particle comes out of the reaction.

\paragraph{}

We expect this phenomena to be particularly significant in samples where the atom that takes part in the reaction can be found not only at the surface of the sample but also in its depth.
Therefore, in implanted samples where the substrate is only a layer, the energy profile will be less evident.

\subsection{Back-scattering}
    \label{sec:elastic_back}

Another possible occurrence is back-scattering, where the incoming proton from the accelerator is elastically reflected back by the sample and subsequently detected.
During this reflection, the energy of the proton and the target is conserved and distributed according to the kinematics of the collision. This will result in a range of energy values for the back-scattered proton.

\subsection{Pile-up}
    \label{sec:pileup}

Pile-up is a phenomenon that is a consequence of the finite time resolution of the detector. It occurs when two particles are detected at the same time and the detector only detects one with the energy doubled. This is a problem because it can lead to the misinterpretation of the data.
Therefore, we expect that, for very intense energy ranges in the spectrum, the pile-up might be significant.
If we have particles emitted with an $[a,b]$ keV energy range, the pile-up will occur in the $[2a,2b]$ keV.

\section{Experimental Procedure}
    \label{sec:experimentalprocedure}

\subsection{Work Plan}
    \label{sec:workplan}

Obtaining the energy spectrum of the particles resulting from the nuclear reactions between the protons and the samples will take the following steps:

\begin{enumerate}
  \item Accelerator Startup - beam energy increase;
  \item Sample Preparation;
  \item Channel-energy calibration of the detection system with a well-known triple $\alpha$ source;
  \item Lenses and electromagnets set up and adjustment;
  \item Obtaining the energy spectrum of all samples;
\end{enumerate}

\subsection{Experimental Set-Up}
    \label{sec:setup}

The experimental set-up consists of a Tandem accelerator, a beam line with electromagnets and a chamber with both the samples and the detectors.
Theoretically, the accelerator produces a beam of protons with a kinetic energy of up to 5 MeV, which is deviated by the steering electromagnet before hitting the sample. 
The electromagnets deflect particles with different angles according to their charge-to-mass ratio and, with that, allows us to remove any unwanted particles from the beam and only let the protons pass through.
The detectors used are silicon-based and detect charged particles through the ionization they produce in the silicon.
In addition, the detectors are connected to an electronic system that amplifies the signal and sends it to a computer where the spectra can be seen.
The beam line is under a vacuum of around $10^{-7}$ mbar in order to avoid the protons colliding with other particles and losing energy. 
They reach the sample with the specified kinetic energy and cause a nuclear reaction whose products are the charged particles detected. 
We aim to analyze the energy spectrum of these particles.

\begin{center}
    \label{TT_21}
    \centering
    \includegraphics[scale = 0.15]{../images/scheme.jpeg} 
    \captionof{figure}{Experimental Set-up}
\end{center}

\subsection{Beam Energy Selection}
    \label{sec:beamenergyselection}

In order to choose the energy of the beam we need to take into account the cross-section of the reaction between the protons and the Li-7. We collected data from the IBANDL database for this reaction in a range of [0.5,7] MeV and plotted it in the following graph:

\begin{center}
    \label{TT_21}
    \centering
    \includegraphics[scale = 0.17]{../images/Li_crosssection_energy.jpeg}
    \captionof{figure}{Cross-section of the reaction between protons and Li-7 as a function of the beam energy}
\end{center}

We are only interested in the energy range of [0.5,5] MeV because the energy of the beam is limited by the accelerators range. As we can see in the above figure, the cross-section of the reaction is maximum in the [2.5, 3] MeV range. Therefore, since we want to detect the Li-7 in the samples, we should choose the energy of the beam to be in this range.
\paragraph{}
However, practically, the accelerator cannot reach a beam within the [2.5, 3] MeV energy range. Increasing beam energies take exponentially more time to be set up. Also, we are interested in the possible boron identification in some samples. Boron does not react with such high energies but a lower energy beam is effective in doing that.
Having this into consideration, we choose a beam energy of $1.3$ MeV for our experimental procedure.

\subsection{NRA Calculations - Theoretical Values}
    \label{sec:theoreticalvalues}

After concluding that the best beam energy to work within the previously described conditions is $1.3$ MeV, we now need to obtain theoretical values for their energy. In order to do this, we will use the NRA calculator \cite{NRAEnergyCalc}. This is an online calculator developed by investigators at CTN that allows researchers to compute the $Q$ value of a reaction as well as the energy values and scattering angles of the particles these reactions emit. Considering an incident beam whose protons have $E_p = 1.3$ MeV and a scattering angle of $\theta = 165 ^{\circ}$ of our detector we computed the following values:

\paragraph{Lithium} For the reaction \ref{eq:reactionLi}:
\begin{equation}
\begin{split}
  E_{\alpha_{0_1}} &= 7.62980 \text{ MeV} \\
  E_{\alpha_{0_2}} &= 11.03144 \text{ MeV} \\
  \theta_{\alpha_{0_1}} &= 165 ^{\circ} \\
  \theta_{\alpha_{0_2}} &= 12.4 ^{\circ} \\
  Q (\ce{^7Li}) &= 17.346 \text{ MeV} \\
\end{split}
\label{eq:energiesli}
\end{equation}

$E_{\alpha_{0_1}}$ and $E_{\alpha_{0_2}}$ are the energies of both alpha particles considering that they are in the fundamental state. 
In this case, the only other option would be for the $\ce{^8Be}$ to be formed in an excited state and therefore give origin to alpha particles also in an excited state. 
There is not enough energy for this to happen since the difference between the ground state and the first state in a helium nucleus is very big.
\paragraph{}
Also, we can see that the scattering angle of the second alpha particle is around $12 ^{\circ}$ meaning that it is different from the detector's angle and will not be detected. To sum up, if there is lithium in any sample that reacts with the incoming protons according to \ref{eq:reactionLi}, we expect to detect alpha particles that were emitted with $7.6298 \text{ MeV}$.

% \paragraph{Boron} Secondly, for the reaction \ref{eq:reactionB}:
% \begin{equation}
% \begin{split}
%   E_{\alpha_0} &= 3.749 \text{ MeV} \\
%   E_{\ce{^8Be}} &= 3.1116 \text{ MeV} \\
%   \theta_{\alpha_0} &= 165 ^{\circ} \\
%   \theta_{\ce{^8Be}} &= 11.6 ^{\circ}\\
%   Q (\ce{^11Br}) &= 8.591 \text{ MeV} \\
% \end{split}
% \label{eq:energiesBr}
% \end{equation}

% We need to comment these energies.

\subsection{Procedure}
    \label{sec:procedure}

We started by displaying the samples in a vertical sample holder with the given order: in the lowest position the implanted sample, followed by lithium aluminate ($\ce{LiAlO2}$) and finally lithium fluoride ($\ce{LiF}$). 
In order to do a calibration we added a triple $\alpha$ source on the holder above the lithium fluoride. 
Using a pachymeter, we measured the distance between the samples so that afterwards we could control the height of the holder in the chamber and change the beam's target.
We placed the sample holder on a support and inserted both of them inside the chamber. 
A vacuum machine was connected to the accelerator to reduce the pressure inside both the chamber and the beam.
Afterwards, we started the procedure to turn on the TANDEM accelerator, according to the provided instruction manual. 
Firstly, we turned on the $\ce{H^+}$ duoplasmatron source of the accelerator as well as the magnets and lenses. 
Then, we opened the valves to propagate the vacuum throughout the entirety of the tubes.
In the control panel, we adjusted the angles of the lenses and the magnets current according to the manual, so that the beam would reach the samples. 
Finally, we used a collimator with a 2mm radius to collimate the beam.
In order to collect the data, we used a computer program to obtain the spectra of the different sources. 
The multichannel analyzer (MCA) used to acquire data was gradually set to 70 V and the angle between the top two detectors was $165 ^{\circ}$.

\section*{Analysis}

\section{Calibration}
    \label{sec:calibration}

For the calibration, we used a triple $\alpha$ source.
In order to do this procedure, for each of spectrum relevant peaks, we need to find the channel with the maximum number of counts.
Then, we find the theoretical values for these peaks and, using both the experimental values for the peaks in channel units and the theoretical ones in MeV, we want to obtain a mathematical relation that allows us to convert channel values into energy values.
To accomplish this, we will make a linear regression with the energy-channel data.

The spectrum obtained for the triple source was the following:
%Ea quis voluptate nisi culpa. Pariatur ad nostrud magna et. Occaecat tempor deserunt enim enim Lorem fugiat magna sint. Occaecat duis occaecat dolore cupidatat qui laboris reprehenderit. Dolor mollit sit tempor labore ad veniam adipisicing duis magna et duis.
%Sunt nostrud in dolor tempor minim velit incididunt Lorem adipisicing ullamco ad qui ad. Veniam magna anim Lorem in ad id qui Lorem ex cillum. Laboris cillum ad anim dolore non aute commodo. Nulla commodo reprehenderit mollit adipisicing do cupidatat proident. Adipisicing eu id incididunt tempor ad fugiat ullamco sit aliqua nulla ut duis labore nostrud. Nostrud deserunt excepteur ex et adipisicing nostrud sint dolor laboris nisi reprehenderit esse eu. Ex non fugiat mollit exercitation minim enim id ipsum velit reprehenderit ea proident nisi.
\begin{center}
    \label{TT_21}
    \centering
    \includegraphics[scale = 0.3]{../../images/Calib.png}
    \vspace{-5mm}
    \captionof{figure}{Spectrum of the Calibration source}
\end{center}

We separated the spectra of
detectors 0, 1 and 2 (2 corresponding to the bottom one) and used a triple gaussian sum function to fit on the three energy peaks. According to theory, the first peak corresponds to the decay of plutonium into uranium, the second of americium into neptunium and the third of curium into plutonium. The decay equations are:

\begin{equation}
    \ce{ ^{239}Pu -> ^{235}U + \alpha}
\end{equation}
\begin{equation}
    \ce{ ^{241}Am -> ^{237}Np + \alpha}
\end{equation}
\begin{equation}
    \ce{ ^{244}Cm -> ^{240}Pu + \alpha}
\end{equation}

%The equation used for the triple gaussian fit is:
%\begin{equation}
%    \begin{split}
%    f(x) = a_0\exp{\left(-\frac{(x-\mu_0)^2}{\sigma_0}\right)} &+ a_1\exp{\left(-\frac{(x-\mu_1)^2}{\sigma_1}\right)} + \\
%    &+ a_2\exp{\left(-\frac{(x-\mu_2)^2}{\sigma_2}\right)} + c
%    \end{split}
%    \label{eq:califit}
%\end{equation}


We decided to do the analysis only with detector 0. This analysis could be further corroborated using detector 1 data.
Using the triple gaussian, we obtained the following fit for this detector:

\begin{center}
    \label{TT_21}
    \centering
    \includegraphics[scale = 0.5]{../../images/TT_21_Chn0.jpeg}
    \captionof{figure}{Detector 0 spectrum fit with triple gaussian}
\end{center}

%\begin{center}
%    \label{TT_21}
%    \centering
%    \includegraphics[scale = 0.6]{images/TT_21_Chn1.jpeg}
%    \captionof{figure}{Channel 1 spectrum fit with [\ref{eq:califit}]}
%\end{center}
%
%\begin{center}
%    \label{TT_21}
%    \centering
%    \includegraphics[scale = 0.6]{images/TT_21_Chn2.jpeg}
%    \captionof{figure}{Channel 2 spectrum fit with [\ref{eq:califit}]}
%\end{center}

The fit parameters for the means, $\mu_0$, $\mu_1$ and $\mu_2$ obtained from the triple gaussian fit represent the energy value of each peak in channels. The computed values for these parameters, as well as the theoretical values of the energy, in MeV, for the peaks, are organised in the following tables:

\begin{table}[H]
\centering
\begin{tabular}{|w{0.8cm}|w{3cm}|w{2cm}|}
\hline
Peak & Detector 0 [Chn] & Theoretical Energy [keV] \\ \hline
1\textsuperscript{st} & $ \mu_0 = 557.00 \pm 0.11 $ & 5156.59 \\ \hline
2\textsuperscript{nd} & $ \mu_1 = 593.14 \pm 0.10 $ & 5485.56 \\ \hline
3\textsuperscript{rd} & $ \mu_2 = 627.44 \pm 0.10 $ & 5804.82 \\ \hline
\end{tabular}
\caption{Calibration Values for detector 0}
\label{tab:calibration0}
\end{table}

%\begin{table}[H]
%\centering
%\begin{tabular}{|w{0.8cm}|w{3cm}|w{2cm}|}
%\hline
%Peak & Channel 1 [Chn] & Theoretical Energy [keV] \\ \hline
%1\textsuperscript{st} & $ \mu_0 = 568.52 \pm 0.17 $ & 5156.59 \\ \hline
%2\textsuperscript{nd} & $ \mu_1 = 605.39 \pm 0.17 $ & 5485.56 \\ \hline
%3\textsuperscript{rd} & $ \mu_2 = 640.77 \pm 0.17 $ & 5804.82 \\ \hline
%\end{tabular}
%\caption{Calibration Values for Channel 1}
%\label{tab:calibration1}
%\end{table}
%
%\begin{table}[H]
%\centering
%\begin{tabular}{|w{0.8cm}|w{3cm}|w{2cm}|}
%\hline
%Peak & Channel 2 [Chn] & Theoretical Energy [keV] \\ \hline
%1\textsuperscript{st} & $ \mu_0 = 563.09 \pm 0.28 $ & 5156.59 \\ \hline
%2\textsuperscript{nd} & $ \mu_1 = 600.15 \pm 0.26 $ & 5485.56 \\ \hline
%3\textsuperscript{rd} & $ \mu_2 = 634.56 \pm 0.36 $ & 5804.82 \\ \hline
%\end{tabular}
%\caption{Calibration Values for Channel 2}
%\label{tab:calibration2}
%\end{table}

Using the values in the table we now have to compute a calibration equation for detector 0.
We plotted the theoretical values of the energy of the three peaks as a function of fit parameters:\vspace{-2mm}

\begin{center}
    \label{TT_21}
    \centering
    \includegraphics[scale = 0.3]{../../images/Chn0_calib_with_Backscattering.jpeg}
    \captionof{figure}{Calibration "detector 0"}
\end{center}

The calibration equation for the aforementioned detector is:

\begin{equation}
    \begin{split}
        E[\text{keV}] &= m_0 \times E[\text{Chn}] + b_0 \\
        m_0 &= 9.01 \pm 0.02 \\
        b_0 &= 142.83 \pm 8.99
    \end{split}
    \label{eq:calib0}
\end{equation}

%\begin{center}
%    \label{TT_21}
%    \centering
%    \includegraphics[scale = 0.6]{images/TT_21_Chn1_calib.jpeg}
%    \captionof{figure}{Calibration Channel 1}
%\end{center}
%
%The calibration equation for channel 1 is:
%
%\begin{equation}
%    \begin{split}
%        E[\text{keV}] &= m_1 \times E[\text{Chn}] + b_1 \\
%        m_1 &= 8.97 \pm 0.03 \\
%        b_1 &= 55.40 \pm 17.53
%    \end{split}
%    \label{eq:calib1}
%\end{equation}
%
%\begin{center}
%    \label{TT_21}
%    \centering
%    \includegraphics[scale = 0.6]{images/TT_21_Chn2_calib.jpeg}
%    \captionof{figure}{Calibration Channel 2}
%\end{center}
%
%The calibration equation for channel 2 is:
%
%\begin{equation}
%    \begin{split}
%        E[\text{keV}] &= m_2 \times E[\text{Chn}] + b_2 \\
%        m_2 &= 9.07 \pm 0.12 \\
%        b_2 &= 48.17 \pm 69.69
%    \end{split}
%    \label{eq:calib2}
%\end{equation}
%

\section{Alpha Peaks}
    \label{sec:alphapeaks}

For the analysis, we began by plotting the theoretical energy values of the two alphas.

\begin{center}
    \label{alfa_peaks}
    \centering
    \includegraphics[scale = 0.28]{../../images/AlphaPeaks.jpeg}
    \captionof{figure}{Theoretical energy values of the two alphas}
\end{center}

The energies obtained were 7629.80 kEv and 11031.44 kEv from section [\ref{sec:theoreticalvalues}].
For the identification of lithium in our samples, we need to compare the theoretical energy values with the experimental ones.
Since we know the emitted alpha particles from our samples are more energetic than their theoretical values, it is expected that the experimental peaks are slightly ahead of the theoretical ones in the spectrum.
If there is a peak in the spectrum that is close to the theoretical value, we can confirm the presence of lithium in the sample.
For the reasons given in section [\ref{sec:theoreticalvalues}], due to the detector's geometry, we expect to only detect $\alpha_0$.

\section{Lithium Fluoride Sample}
    \label{sec:lif}

For the LiF sample, we used the calibration [\ref{sec:calibration}] to convert the channel values into energy values.
We overlapped the expected $\alpha$ energy with the spectrum, and obtained the following:

\begin{center}
    \centering
    \includegraphics[scale = 0.3]{../../images/OverlapLiF_NC.jpeg}
    \captionof{figure}{Spectrum of the $\ce{LiF}$ sample}
\end{center}

The experimental $\alpha_0$ is extremely close to the theoretical value, therefore we can confirm the presence of lithium in this sample.
Moreover, it is according to the expected hypothesis that the observed lithium energy is slightly higher than the theoretical value.
% Analysing this spectrum further:

\begin{center}
    \label{TT_21}
    \centering
    \includegraphics[scale = 0.3]{../../images/FullAnalysisLiF.jpeg}
    \captionof{figure}{Full Analysis of $\ce{LiF}$ Spectrum}
\end{center}

There are three noticeable structures in this spectrum.
The first one is a peak around 0 keV, which is just electronic noise. The following structure is back scattering [\ref{sec:elastic_back}], identified by the barrier of 1146.5 kEv.
The back scattering is followed by event pile-up [\ref{sec:pileup}].
The last structure is the lithium peak. The depth of the sample creates the visible energy profile and this procedure is explained in section [\ref{sec:energy_profile_peak}].

\section{Lithium Aluminate Sample}
    \label{sec:lialo2}

We used the calibration again to obtain the Lithium Aluminate energy spectrum overlapped with the expected $\alpha$ energy:

\begin{center}
    \label{TT_21}
    \centering
    \includegraphics[scale = 0.3]{../../images/OverlapLiAlO2.jpeg}
    \captionof{figure}{Spectrum of the $\ce{LiAlO_2}$ sample}
\end{center}

The experimental $\alpha_0$ is very similar to the theoretical value, hence we can confirm the presence of lithium in this sample.
% If we analyze this spectrum further, we can see the following:

\begin{center}
    \label{TT_21}
    \centering
    \includegraphics[scale = 0.3]{../../images/FullAnalysisLiAlO2.jpeg}
    \captionof{figure}{Full Analysis of $\ce{LiAlO_2}$ Spectrum}
\end{center}

It is evident that the structure of the spectrum follows a similar pattern to the one in the previous sample.
First, there is electrical noise, followed by back scattering [\ref{sec:elastic_back}] and event pile-up [\ref{sec:pileup}].
The back scattering is identified by the barrier of 1150.0 keV.
Finally, there is the lithium peak, which has an energy profile [\ref{sec:energy_profile_peak}] as well due to the depth of the sample.

\section{Implanted Sample}
    \label{sec:implanted}

Finally, we used the calibration to obtain the implanted sample energy spectrum overlapped with the expected $\alpha$ energy:

\begin{center}
    \label{TT_21}
    \centering
    \includegraphics[scale = 0.3]{../../images/OverlapImplanted.jpeg}
    \captionof{figure}{Spectrum of the implanted sample}
\end{center}

There is a peak in the proximity of the $\alpha_0$ peak, which is the expected lithium peak.
One more time we can confirm the presence of lithium in this sample.
% If we analyze the spectrum further:

\begin{center}
    \label{TT_21}
    \centering
    \includegraphics[scale = 0.3]{../../images/FullAnalysisImplanted.jpeg}
    \captionof{figure}{Full Analysis of the Implanted Sample Spectrum}
\end{center}

Just like before, there is electrical noise, followed by back scattering [\ref{sec:elastic_back}] and event pile-up [\ref{sec:pileup}].
However, this sample is implanted only on the surface, therefore there is no energy profile as expected, since the lithium isn't being detected at different depths.

\section{Quantitative Analysis}
    \label{sec:quantitative}

According to \cite{chu1978backscattering} it is possible to calculate the concentration of the element in the samples using with the following equations:

\begin{minipage}[t]{0.5\linewidth-0.5em} % some separation
    \vspace{0pt} % anchor for [t]
    \vspace{\dimexpr\ht\strutbox-\topskip}% remove excess glue
    \begin{equation}
        N\tau_i=\frac{H_i}{\sigma (E_i)\Omega Q}
    \end{equation}
\end{minipage}\hfill
\begin{minipage}[t]{0.5\linewidth-0.5em} % some separation
    \vspace{0pt} % anchor for [t]
    \begin{equation}
        N\Delta x=\frac{A}{\sigma (E_i)\Omega Q}
    \end{equation}
\end{minipage}

\vspace{4mm}

Where $N\tau_i$ are $N\Delta x$ different values for the superficial concentration of the element. For comparison, we can approximate $N\Delta x \approx N\tau_i \times$ \#channels.

$H_i$ is the height, in counts, on a spectrum with accumulation (used in the $\ce{LiAlO2}$ and $\ce{LiF}$ samples), $A$ is the area of a singular peak (used in the implanted sample), $\sigma (E_i) $ is the cross section of the reaction, $\Omega$ is the solid angle of the detector and $Q$ is the max charge acquisition divided by $e$.\\

With this we get:

Implanted Sample: $N\Delta x = 9.25\times 10^{16}$ atoms per cm$^2$\\
$\ce{LiAlO2}$: $N\tau_i=4.31\times10^{17}$ atoms per cm$^2$\\
$\ce{LiF}$: $N\tau_i=1.15\times10^{18}$ atoms per cm$^2$

For comparison:
$\ce{LiAlO2}$: $N\Delta x=2.15\times10^{19}$ atoms per cm$^2$\\
$\ce{LiF}$: $N\Delta x=5.71\times10^{19}$ atoms per cm$^2$\\

From this, it is noticeable that the implanted sample has around 200x less Lithium than the $\ce{LiAlO2}$ and 600x less Lithium than the $\ce{LiF}$, as expected.

\section{Conclusion}
    \label{sec:conclusion}

To sum up, we were able to successfully conduct the proposed research project.
Nuclear reactions can be a very effective technique to detect the presence of elements in samples, despite its high cost.
We were able to detect the presence of lithium in all the samples. 
Furthermore, we managed to go beyond our proposal and estimate the amount of lithium in each sample.
% The results obtained were in accordance with the values obtained from the Danish Institute that provided the samples. MEIO INVENTADO

\nocite{*}
\printbibliography


\end{multicols}

\end{document}