
% RUN in terminal (without bibliography):
% pdflatex -output-directory=/Users/salvatorpes/Desktop/LFEUI/text/proposta /Users/salvatorpes/Desktop/LFEUI/text/proposta/prop.tex

% limpar:
% latexmk -c

\documentclass{article}

\author{
    \begin{tabular}{rl}
        Estêvão Gomes (ist1102650) & Sofia Nunes (ist1102633) \\
        Pedro Curvo (ist1102716) & Salvador Torpes (ist1102474)
    \end{tabular}
}

\usepackage[utf8]{inputenc}
\usepackage[portuguese]{babel}
\usepackage[letterpaper,top=10mm,bottom=15mm,left=10mm,right=10mm,marginparwidth=1.75cm]{geometry}
\usepackage{multicol}
\usepackage{graphicx}
\usepackage{subcaption}
\usepackage{tabularx}
\usepackage{booktabs}
\usepackage{array}
\usepackage{makecell}
\usepackage{titlesec}
\usepackage{multirow}
\usepackage{amsmath}
\usepackage{makecell}
\usepackage{url}
\usepackage{csquotes}
\usepackage{caption}
\usepackage{enumitem}
\usepackage{textcomp}
\usepackage{pdflscape}
\usepackage{makeidx}
% \usepackage{tocbibind}
\providecommand{\tightlist}{\relax}
\usepackage{tocloft}
\renewcommand{\cftsecindent}{0em}
\renewcommand{\cftsubsecindent}{1em}
\renewcommand{\cftsecfont}{\bfseries}
\renewcommand{\cftsubsecfont}{\itshape}
\setlength{\cftsubsecnumwidth}{0em}

\usepackage[version=4]{mhchem}
\usepackage{hyperref} % Remove "pdftex" option here
\usepackage{float}
\usepackage{fancyhdr}
\usepackage{ragged2e}
\usepackage{xkeyval}
%\usepackage{minted}
%\usemintedstyle{manni}
\usepackage{listings}
\usepackage{amssymb}


\usepackage{xcolor}
\usepackage{tikz}

\usetikzlibrary{positioning}
\usetikzlibrary{positioning, arrows.meta}
\usepackage{adjustbox}
\usepackage{sidecap}
\usepackage{graphicx}

\usepackage{tikz-3dplot}
\usepackage{pgfplots}
\usetikzlibrary{calc, 3d, arrows}



\usetikzlibrary{shapes.geometric, arrows}


\lstset{
    language=Python,
    basicstyle=\ttfamily,
    keywordstyle=\color{blue},
    commentstyle=\color{gray},
    stringstyle=\color{orange},
    numbers=left,
    numberstyle=\tiny,
    numbersep=5pt,
    showspaces=false,
    showstringspaces=false,
    breaklines=true,
    frame=tb,
    framexleftmargin=2em,
    xleftmargin=2em,
}


%\usepackage{fontspec}

%\setmonofont{Fira Code}

\fancyhf{}
\cfoot{\thepage}
\fancyhf{} % Clear all header and footer fields
\renewcommand{\headrulewidth}{0pt} % Remove the header rule line
\cfoot{\thepage} % Set the page number in the center of the footer

\pagestyle{fancy} % Apply the fancy page style

\setlength\columnsep{20pt}

\renewcommand{\familydefault}{\sfdefault}

\newenvironment{Figure}
  {\par\medskip\noindent\minipage{\linewidth}}
  {\endminipage\par\medskip}

\makeatletter
\newenvironment{figurehere}
{\def\@captype{figure}}
{}
\makeatother

\hypersetup{
  colorlinks,
  linkcolor=blue,
  anchorcolor=black,
  citecolor=cyan,
  filecolor=cyan,
  menucolor=cyan,
  urlcolor=cyan,
  bookmarksopen=true,
  bookmarksnumbered=true
}

\makeindex


\title{\vspace{-13mm}\includegraphics[width=15mm,scale=3]{../images/IST_Logo.png}\\ \vspace{5mm}
LFEUI - Notas Teóricas \vspace{-5mm}}
\date{30 Novembro 2023}

\usepackage{sansmathfonts}
\usepackage[T1]{fontenc}
\usepackage[OT1]{fontenc}

\titleformat{\section}{\normalfont\large\bfseries}{\thesection}{1em}{}



\usepackage[style=numeric]{biblatex} % Choose your desired citation style
\addbibresource{../references/prop.bib} % Specify your .bib file

\begin{document}

\renewcommand{\arraystretch}{1.5}
\setlength{\columnseprule}{0.4pt}
\tdplotsetmaincoords{70}{110} % Set the viewing angle
\newcolumntype{M}[1]{>{\centering\arraybackslash\vspace{#1}}m{0.5\linewidth}<{\vspace{#1}}}
\newcolumntype{C}[2]{>{\centering\arraybackslash\vspace{#1}\rule{0pt}{#1}\hspace{0pt}}m{#2}}
\newcolumntype{w}[1]{>{\centering\arraybackslash}m{#1}}

\renewcommand*\familydefault{\sfdefault} %% Only if the base font of the document is to be sans serif

\maketitle

\vspace{-5mm}

\hrulefill

\begin{multicols}{2}

\section{Acelerador de Partículas}

O acelerador utilizado possui no seu interior um gás.

\section{Lítio}

Vamos trabalhar com o lítio, um elemento altamente reativo, maioritariamente através de reações de oxidação com o vapor de água. 
A solução para trabalhar com este elemento é usar fontes inertes - uma fonte inerte de lítio consiste em lítio no interior de duas camadas de vidro.
O vidro é composto por silicatos de boro.

\section{Objetivo}

O objetivo é obter o espetro de emissão de duas fontes diferentes, ambas contendo lítio. 

\section{Reação}

A reação que pretendemos estudar envolve o choque de um átomo de lítio 7 com um protão, que resulta na formação de um átomo de berílio 8 que decai para duas partículas alfa (núcleo de hélio 4) - o decaimento é tão rápido que não é detetado.

\begin{equation}
    \ce{^7Li + ^1H -> ^8Be -> 2\alpha = 2^4He}
\end{equation}

Esta reação é idêntica para as duas amostras que vamos estudar.

\subsection*{Valor $Q$ da reação}

O valor $Q$ da reação é a energia libertada na reação - é a energia que as partículas alfa possuem quando são emitidas.
Calcula-se a partir da diferença de massa entre os reagentes e os produtos da reação:

\begin{equation}
  \begin{split}
      Q &= \Delta m c^2 \\
      &= c^2 \left( m_{\text{reagents}} - m_{\text{products}} \right) \\
      &= c^2 \left( m_{\ce{^7Li}} + m_{\ce{^1H}} - m_{\ce{^8Be}} \right)
  \end{split}
\end{equation}
    

\section{Processo Experimental}

Queremos obter o espetro energético para perceber qual a energia com que as partículas são emitidas

\section{Montagem Experimental}

Temos um acelerador Tandem de 2.5 MeV, que acelera protões - temos 2 tubos de aceleração em paralelo o que faz com que a tensão à qual os partículas protões são aceleradas possa chegar aos 5 MV. Os dois tubos estão alinhados o que faz com que o protão seja 'duplamente' acelerado.
Acelerar um protão a 5 MeV significa que a variação de energia cinética do eletrão desde o íncio do tubo (em repouso) até ao seu fim é de 5 MeV - este é o valor da diferença de potencial elétrico entre as duas extremidades/terminais do tubo de aceleração. 
Este acelerador possui um elevador de tensão - associação de condensadores e resistências que permitem aumentar a tensão.

\paragraph*{Vácuo} O feixe propaga-se ao longo dos dois tubos de aceleração que se encontra sobre um vácuo de $10^{-7}$ mbar. O vácuo é necessário para que as partículas não colidam com moléculas de ar e percam energia.
Este vácuo é bastante bom sendo possível que existam colisão no caminho apesar de em média o número destas colisões é nulo.
A este nível de pressão ($10^{-7}$ mbar), existem cerca de $10^9$ moléculas de ar por cm$^3$. Na atmosfera normal (pressão de 1 bar) existem $10^{19}$ moléculas de ar por cm$^3$ - no vácuo intergalático temos 5 moléculas por cm$^3$ - apesar de o vácuo deste vácuo não ser comparável ao do espaço, é um vácuo bastante bom. 
Para estimar estas quantidades de partículas por cm$^3$ consideramos que a diferença de pressão entre a atmosfera normal e o vácuo e consideramos que o volume de um mol de gás é $22.4$ L/mol na atmosfera normal - isto implica que há uma mol de moléculas por litro - $6.02 \times 10^{23}$ moléculas por litro.

\paragraph*{Eletroímane} No nosso eletroímane, partículas com diferentes racios carga/massa têm ângulos de deflexão diferentes - o ângulo de deflexão é inversamente proporcional ao racio carga/massa. Este elemento é responsável por fazer com que todas as partículas que não os protões que se encontrem no feixe sejam desviadas para fora do feixe e por isso não atingem o sensor. É por isto que existe um grande benefício em ter um detetor não linear - fitração do feixe de partículas.

\paragraph*{Detetores} Vamos usar detetores de silício - a energia de uma partícula incidente no detetor é calculada a partir do número de pares electrão-buraco que são criados no detetor devido à sua passagem.

\paragraph*{Eletrónica} Existe um sistema de eletrónica digital responsável por fazer a leitura dos detetores e por enviar os sinais para o computador onde podemos observar os espetros obtidos.

\section{Amostras}

A atividade envolve o estudo dos espetros provenientes de duas amostras diferentes, ambas contendo lítio - uma amostra que é apenas um vidro composto por silicatos e óxidos e outra que é uma amostra com lítio implantado em silício.

\paragraph*{Implantação} A implantação é o processo de formacação de amostras implantadas - para obter uma amostra de lítio 7 implantado em silício usou-se um feixe de lítio 7 com potência na ordem dos kilovolts - quanto maior a potência do feixe, maior é a profundidade a que o lítio é implantado no silício. 
Existe um fórmula que nos permite saber a profundidade a que o lítio é implantado no silício em função da potência do feixe de lítio 7 e da densidade do silício.

\paragraph*{Diferenças} O objetivo de estudar estas duas amostras é avaliar a diferença de espetros de emissão de partículas alfa provenientes de lítio 7 em diferentes profundidades.

\begin{table}[H]
\centering
\begin{tabular}{|w{0.2\linewidth}|w{0.7\linewidth}|}
\hline
Amostra & Descrição \\ \hline
Vidro Normal & Amostra formada por vidro normal com silício, lítio, boro e talvez outros resíduos - o lítio aparece na forma de óxido de lítio de modo a que seja pouco reativo. Esta amaostra é homogénea na sua constituição ao longo de toda a sua espessura uma vez que o vidro é formado diretamente com os elementos. \\ \hline
Implantada & Amostra obtida através da implatação de uma camada de lítio numa amostra de sílicio (pura) - o que obtemos é uma amostra que nos primeiros nanómetrod e espessura possui lítio e depois apenas possui silício. \\ \hline
\end{tabular}
\end{table}

\section{Resultados Esperados}

\paragraph*{Perfis} Um perfil é uma parte contínua do espetro na zona energética anterior aos picos de partículas alfa.
Esperamos ver isto na amostra de vidro devido às partículas alfa que provém de diferentes profundidades da amostra de vidro.
Quanto maior for a profundidade, como estamos a falar de partículas carregadas e não de fotões, maior é a energia dissipada para o material durante o percurso. 
Para partículas emitidas com a mesma energia, quanto maior for a profundidade, menor é a energia com que a partícula é detetada - cria-se um perfil no espetro devido a isto. 
Como na amostra implantada o lítio está apenas presente numa camada superior muito fina, não esperamos obter um perfil no espetro.

\paragraph*{RBS - Rutherford Backscattering} é um fenómeno no qual as partículas alfa são retrodispaersas para outras direções e por isso possuem energias diferentes. 
A maior parte das partículas alfa são detetadas com o valor de energia esperada mas podemos ter outras zonas do espetro onde tenhamos partículas alfa com energias diferentes e dispersas devido a este fenómeno.

\paragraph*{Elastic Backscattering} Isto acontece quando um dos protões do feixe incidente é refletido elasticamente na superfície da amostra - o protão é refletido com a mesma energia com que incidiu na amostra e continua a trajetória na mesma direção mas com sentido oposto.

\section{Utilidade}

\begin{enumerate}
  \item Fusão Nuclear - lítio 6 para produção de neutrões -  usam-se materiais muito enriquecidos em lítio 6;
  \item Baterias - lítio 7;
  \item Indústria Farmaceûtica - lítio é utilizado em vários medicamentos;
\end{enumerate}

\section{Fenómenos}

Existem 3 fenómenos diferentes que acontecem nas nossas amostras quando os seus atómos colidem com os protões do feixe: reação nuclear, RBS e elastic backscattering.
Cada um destes 3 fenómenos possui uma secção eficaz diferente - a secção eficaz é uma medida da probabilidade de um dado fenómeno acontecer.
A secção eficaz do RBS e do elastic backscattering são maiores do que a secção eficaz da reação nuclear - isto significa que a probabilidade de um protão colidir com um átomo da amostra e ser refletido é maior do que a probabilidade de um protão colidir com um átomo da amostra e reagir com ele.
Nos espetros detetados podemos ver uma 'bossa' inicial que corresponde ao RBS e ao elastic backscattering e depois vemos um pico que corresponde à reação nuclear do lítio 7 com o protão.
A seguinte fórmula é válida para a geometria da nossa experiência, sendo apenas necessário aplicá-la para cada uma das 3 secções eficazes:

\[ A = \left( \frac{q_{\text{beam}}}{e} \right) \Omega \overline{\sigma} (N \Delta x) \]

\section{Pico de Reação Nuclear}

Existem 2 picos teóricos esperados no espetro devido à reação nuclear entre o lítio e o protão.
O feixe que utilizamos na atividade experimental possui uma energia de 1.3 MeV - a zona de energia dos protões que maximiza a intensidade deste pico é de $[2.5,3]$ MeV, no entanto, não conseguimos alcançar estas energias com o nosso acelerador (apesar de teoricamente serem possíveis). 
Para além disto estamos também interessados em detetar uma possível presença de boro nas amostras e o boro não reage com protões de tão elevada energia - assim escohemos uma energia onde tanto o lítio como o boro reagem com os protões: 1.3 MeV.

\paragraph*{Cálculo do Valor}

Recorremos ao NRA calculator para calcular o valor dos picos de reação nuclear para o lítio 7 e para o boro 11.
Usamos 1.3 MeV para os protões incidentes, 165 graus para o ângulo de scattering do detetor e obtivemos os seguintes valores:

\begin{equation}
\begin{split}
  E_{\alpha_{0_1}} &= 7.6298 \text{ MeV} \\
  E_{\alpha_{0_2}} &= 11.0314 \text{ MeV} \\
  \theta_{\alpha_{0_1}} &= 165 \\
  \theta_{\alpha_{0_2}} &= 12 \\
  Q (\ce{^7Li}) &= 17.346 \text{ MeV} \\
\end{split}
\label{eq:energiesli}
\end{equation}

Para o lítio não temos partículas $\alpha_1$ uma vez que: $\alpha_0$ são partículas alfa no estado fundamental e $\alpha_1$ são partículas alfa no estado excitado - a energia que temos no feixe não é suficiente para excitar as partículas alfa e por isso elas apenas são emitidas no estado fundamental.
Os átomos de hélio 4 são ultra estáveis e o primeiro estado excitado está muito acima do estado fundamental - a energia necessária para excitar as partículas alfa é muito elevada e por isso não temos partículas alfa no estado excitado.
\paragraph{} Note-se também que o ângulo de scattering para a segunda partícula alfa $\alpha_{0_2}$ é de 12 graus, ângulo no qual não há detetor, logo, apenas vamos detetar um pico de partículas em 7.6298 MeV.

\paragraph*{} Para o boro temos:

\begin{equation}
\begin{split}
  E_{\alpha_0} &= 3.7372 \text{ MeV} \\
  E_{\ce{^8Be}} &= 3.1086 \text{ MeV} \\
  \theta_{\alpha_0} &= 165 \\
  \theta_{\ce{Be8}} &= 11 \\
  Q (\ce{^11Br}) &= 8.591 \text{ MeV} \\
\end{split}
\label{eq:energiesBr}
\end{equation}


\section{Back Scattering - Dúvidas}

o backscattering é todo elástico e um caso particular deste é o de rutherford.
O caso particular de rutherford considera os atomos como esferas perfeitas para calcular a secção eficaz deste fenomeno.
Em geral, as secções eficazes do backscattering obtém-se experimentalmente e não usamos as de rutherford.
O que interessa é que o backscattering elastico como fenomeno e não as suas diferentes secções eficazes.
Pileup - o detetor recebe duas partículas de backscattering em simultaneo e assum que é so uma e por isso soma uma contagem com o dobro da energia.
A particula emitida no backscattering pode ter diferentes valores de energia de acordo com a colisao elastica entre os dois atomos (a mecanica diz nos como calular a energia com que cada átomo, o incidente e o da amosta à superfície, fica depois da colisão) e por isso a soma da energia de duas também pode estar num dado intervalo.
O pico inicial é do ruido eletronico e por isso desconsideramos. Usar detector em vez de channel.
A secção eficaz da reação nuclear com o litio varia com a energia do protão incidente.
O protão incidente perde energia à medida que percorre a amostra devido ao stopping power da amostra em si e por isso a secção eficaz de reacções que ocorrem mais no interior da amostra é diferente das que ocorrem à sua superfície.


\printbibliography
\nocite{*}

\end{multicols}

\end{document}